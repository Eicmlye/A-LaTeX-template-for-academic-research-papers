% version 2023.0528.0943
% PACKAGE {amsthm} OR {amsmath} required.
% PACKAGE {ctex} required.

% \newtheorem usage: 
	%# \newtheorem{TheoremName}[Counter_countWith]{Title}[Counter_recountWhen]
		%% TheoremName: name of the new theorem environment. Redefinition raises error. A new counter ,say Counter_TheoremName, with exactly the same name (so Counter_TheoremName is actually the counter TheoremName) will be built by default, and can be used as Counter_countWith in preceding environments. 
		%% NOTICE that '*' is not detectable in TheoremName. 
		%% Counter_countWith: [optional] DEFAULT: count with Counter_TheoremName; if valued, set Counter_TheoremName to Counter_countWith, and they will share the same counter. 
		%% Title: text message. Set the title of the environment, like 'Theorem '. 
		%% Counter_recountWhen: [optional] DEFAULT: count with the root counter (section in article document, and chapter in book document); if valued, recount when the counter Counter_recountWhen is activated. 
		
% Theorem environment usage: 
	%# \begin{TheoremName}[Subtitle]
	%# Statements
	%# \end{TheoremName}

% PACKAGE {ntheorem} required
\usepackage[thmmarks]{ntheorem}
% \newtheorem usage: 
	%# {
	%# \theorempreskipamount{DEFAULT: \topsep}
		%% set space before the environment
	%# \theorempostskipamount{DEFAULT: \topsep}
		%% set space after the environment
	%# \theoremstyle{DEFAULT: plain}
		%% arguments: 
			%# plain
			%# break: use \\ after title and number
			%# change: print number before title
			%# changebreak
			%# margin: print number in the left margin
			%# marginbreak
			%# nonumberplain: plain with no number
			%# nonumberbreak: break with no number
			%# empty: no title and no number
	%# \theoremheaderfont{DEFAULT: \normalfont\bfseries}
		%% \color{} should be added here
	%# \theorembodyfont{DEFAULT: \itshape}
	%# \theoremseparator{}
		%% set seperator between title and statements
	%# \theoremnumbering{DEFAULT: arabic}
		%% arguments: 
			%# arabic
			%# alph
			%# Alph
			%# roman
			%# Roman
			%# greek
			%# Greek
			%# chinese
			%# fnsymbol
	%# \theoremsymbol{}
		%% Package option [thmmarks] required.
		%% When ended with single-lined equations, \theoremsymbol will not be printed. 
		%% Text message. Set end symbol. 
	%# \newframedtheorem{TheoremName}[Counter_countWith]{Title}[Counter_recountWhen]
		%% Package option [framed] required
		%% PACKAGE {framed} required
		%% Usage the same as \newtheorem, but with frame.
	% \newtheorem{TheoremName}[Counter_countWith]{Title}[Counter_recountWhen]

%% theorem environment
{
\theorembodyfont{\normalfont}
\theoremseparator{\quad}

\newtheorem{defn}{定义}[section]
\newtheorem{exap}[defn]{例}
\newtheorem{propo}[defn]{命题}
\newtheorem{theorem}[defn]{定理}
\newtheorem{lemma}[defn]{引理}
\newtheorem{corol}[defn]{推论}
}
{
\theoremstyle{nonumberplain}
\theorembodyfont{\normalfont}
\theoremseparator{\quad}
\theoremsymbol{$\Box$}

\newtheorem{proof}{证明}
}
{
\theoremstyle{nonumberplain}
\theorembodyfont{\normalfont}
\theoremseparator{\quad}
\theoremsymbol{}

\newtheorem{prf}{证明}
\newtheorem{sol}{解}
\newtheorem{remk}{注}
}