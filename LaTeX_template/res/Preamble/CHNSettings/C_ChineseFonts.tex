% version 2023.0528.0943

\usepackage{ctex}
	%% Do NOT import this package if a full-English article is required. 
	
	%# \zihao{}
	
	%% {ctex} fonts
		%#	\songti 宋体
		%#	\kaishu 楷体
		%#	\heiti 黑体
		%#	\yahei 微软雅黑
		%#	\fangsong 仿宋
		%#	\youyuan 幼圆
		%#	\lishu 隶书
	
\usepackage{xeCJK}
	\setmainfont{Times New Roman}
		%% \textrm{} for locally use main font.
		%% \rmfamily for globally use main font.
	\setsansfont{Verdana}
		%% \textsf{} for locally use sans serif.
		%% \sffamily for globally use sans serif.
	\setmonofont{Courier New}
		%% \texttt{} for locally use mono font.
		%% \ttfamily for globally use mono font.
	\setCJKfamilyfont{zhsong}[AutoFakeBold={2.17}]{SimSun}
	\renewcommand*{\songti}{\CJKfamily{zhsong}}
	\setCJKfamilyfont{zhhei}[AutoFakeBold={2.17}]{SimHei}
	\renewcommand*{\heiti}{\CJKfamily{zhhei}}
	%% If you want to use fake bold font, explicitly write 
	%% the font you want first, e.g. \textbf{\songti 你好}.
	
	%> See detailed font size, shape modifiers at https://blog.csdn.net/dreaming_coder/article/details/115723681.
	
	%#	{\CJKfontspec{Adobe Song Std L}} % Adobe 宋体
	%#	{\CJKfontspec{Adobe Kaiti Std R}} % Adobe 楷体
	%#	{\CJKfontspec{Adobe Heiti Std R}} % Adobe 黑体
	%#	{\CJKfontspec{Adobe Fangsong Std R}} % Adobe 仿宋
	%#	
	%#	{\CJKfontspec{STSong}} % 华文宋体
	%#	{\CJKfontspec{STZhongsong}} % 华文中宋
	%#	{\CJKfontspec{STKaiti}} % 华文楷体
	%#	{\CJKfontspec{STXingkai}} % 华文行楷
	%#	{\CJKfontspec{STXihei}} % 华文细黑
	%#	{\CJKfontspec{STFangsong}} % 华文仿宋
	%#	{\CJKfontspec{STLiti}} % 华文隶书
	%#	{\CJKfontspec{STCaiyun}} % 华文彩云
	%#	{\CJKfontspec{STHupo}} % 华文琥珀
	%#	
	%#	{\CJKfontspec{FZShuSong-Z01S}} % 方正书宋
	%#	{\CJKfontspec{FZHei-B01S}} % 方正黑体
	%#	{\CJKfontspec{FZKai-Z03S}} % 方正楷体

\usepackage{zhnumber}
	%% Add \zhnum as an option for counter labels. 

%\usepackage{type1cm} % Customized font size. EM deprecated.
	%# {\fontsize{fontsize}{baselineskip}\selectfont}



% 字号转换(pt)
	%%	初号42
	%%	小初36
	%%	一号26
	%%	小一24
	%%	二号22
	%%	小二18
	%%	三号16
	%%	小三15
	%%	四号14
	%%	小四12
	%%	五号10.5
	%%	小五9
	%%	六号7.5
	%%	小六6.5
	%%	七号5.5
	%%	小七5