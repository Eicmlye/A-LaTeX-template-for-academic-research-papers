% version 2023.0528.0943

% Must be used in math mode, otherwise the delimiters won't work.

% Text.
	%\newcommand{\bs}{\backslash} % EM defined auto-completion.
	\newcommand{\prm}{^{\prime}}
	\newcommand{\pprm}{^{\prime\prime}}
	\newcommand{\nprm}[1]{^{(#1)}}
	
% Special symbols.
	%% sets
	\newcommand{\z}[2]{\mathbb{Z}_{#1}^{#2}}
	\newcommand{\eusp}[1]{\mathbb{R}^{#1}} % Euclidean space
	
	%% functions
	\newcommand{\mbeta}[2]{\bm{\mathrm{B}}\mparen{#1,#2}} % Math Beta function
	\newcommand{\mgamma}[1]{\bm{\Gamma}\mparen{#1}} % Math Gamma function

% Constants.
	\newcommand{\mathe}{\mathrm{e}}
	\newcommand{\mathi}{\mathrm{i}}

% Brackets with delimiters.
	%% parentheses
	\newcommand{\mparen}[1]{\left(#1\right)} % Math Parenthesis.
		\newcommand{\bparen}[1]{\big(#1\big)}
		\newcommand{\Bparen}[1]{\Big(#1\Big)}
		\newcommand{\bgparen}[1]{\bigg(#1\bigg)}
		\newcommand{\Bgparen}[1]{\Bigg(#1\Bigg)}
	
	%% brackets
	\newcommand{\mbracket}[1]{\left[#1\right]} % Math Brackets.
		\newcommand{\bbracket}[1]{\big[#1\big]}
		\newcommand{\Bbracket}[1]{\Big[#1\Big]}
		\newcommand{\bgbracket}[1]{\bigg[#1\bigg]}
		\newcommand{\Bgbrackt}[1]{\Bigg[#1\Bigg]}
	\newcommand{\lbracket}[1]{\left[#1\right)}
	\newcommand{\rbracket}[1]{\left(#1\right]}
	
	%% braces
	\newcommand{\mbrace}[1]{\left\{#1\right\}} % Math Braces.
		\newcommand{\bbrace}[1]{\big\{#1\big\}}
		\newcommand{\Bbrace}[1]{\Big\{#1\Big\}}
		\newcommand{\bgbrace}[1]{\bigg\{#1\bigg\}}
		\newcommand{\Bgbrace}[1]{\Bigg\{#1\Bigg\}}

	%% special bracket
	\newcommand{\mset}[2]{\mbrace{\left.#1\middle|#2\right.}} % Math Set.
		\newcommand{\bset}[2]{\bbrace{#1\big|#2}}
		\newcommand{\Bset}[2]{\Bbrace{#1\Big|#2}}
		\newcommand{\bgset}[2]{\bgbrace{#1\big|#2}}
		\newcommand{\Bgset}[2]{\Bgbrace{#1\Big|#2}}
	\newcommand{\mar}[1]{\left\langle#1\right\rangle} % Math Arrow.
		\newcommand{\bmar}[1]{\big\langle#1\big\rangle}
	\newcommand{\floor}[1]{\left\lfloor#1\right\rfloor}
	\newcommand{\ceil}[1]{\left\lceil#1\right\rceil}
	\newcommand{\dbr}[2]{\left\{\begin{array}{#1}#2\end{array}\right.} % Displaystyle brackets.
		%% \begin{numcases}{...}...\end{numcases} is an unaligned numbered alternative for \dbr.
	
	%% absolute value
	\newcommand{\babs}[1]{\big|#1\big|}
	\newcommand{\Babs}[1]{\Big|#1\Big|}
	\newcommand{\bgabs}[1]{\bigg|#1\bigg|}
	\newcommand{\Bgabs}[1]{\Bigg|#1\Bigg|}
	
% displaystyle symbols
	%% displaystyle
	\newcommand{\dpl}{\displaystyle}

	%% limits
	\newcommand{\dlim}[1]{\dpl\lim_{#1}}
	\newcommand{\dlimsup}[1]{\dpl\limsup_{#1}}
	\newcommand{\dliminf}[1]{\dpl\liminf_{#1}}
	\DeclareMathOperator*{\limsOPR}{\overline{\lim}} % aux
	\newcommand{\lims}[1]{\limsOPR_{#1}}
	\newcommand{\dlims}[1]{\dpl\limsOPR_{#1}}
	\DeclareMathOperator*{\limiOPR}{\underline{\lim}} % aux
	\newcommand{\limi}[1]{\limiOPR_{#1}}
	\newcommand{\dlimi}[1]{\dpl\limiOPR_{#1}}
	
	%% bounds
	\newcommand{\dsup}[1]{\dpl\sup_{#1}}
	\newcommand{\dinf}[1]{\dpl\inf_{#1}}
	\newcommand{\dmax}[1]{\dpl\max_{#1}}
	\newcommand{\dmin}[1]{\dpl\min_{#1}}
	
	%% operations
	\newcommand{\dcup}[2]{\dpl\bigcup_{#1}^{#2}}
	\newcommand{\dcap}[2]{\dpl\bigcap_{#1}^{#2}}
	\newcommand{\dsum}[2]{\dpl\sum_{#1}^{#2}}
	\newcommand{\scyc}{\dsum{cyc}{}}
	\newcommand{\dpro}[2]{\dpl\prod_{#1}^{#2}}
	\newcommand{\dopl}[2]{\dpl\oplus_{#1}^{#2}}
	\newcommand{\dcop}[2]{\dpl\coprod_{#1}^{#2}}
	
	%% integrals
	\newcommand{\dint}[2]{\dpl\int_{#1}^{#2}}
	\newcommand{\diint}[1]{\dpl\iint_{#1}}
	\newcommand{\diiint}[1]{\dpl\iiint_{#1}}
	\newcommand{\doint}[1]{\dpl\oint_{#1}}
	
	%% vector analysis
	\DeclareMathOperator{\grad}{grad}
	\DeclareMathOperator{\diver}{div}
	\DeclareMathOperator{\rot}{rot}
	\DeclareMathOperator{\curl}{curl}

% Analysis.
	%% differentiation
	\DeclareMathOperator{\sgn}{sgn}
	\newcommand{\difc}{\Delta} % Difference.
	\newcommand{\diff}{\mathrm{d}} % Differential.
	\newcommand{\deri}[2]{\dfrac{\diff{#1}}{\diff{#2}}} % Derivative.
	\newcommand{\pder}[2]{\dfrac{\partial#1}{\partial#2}} % Partial derivative.
	\DeclareMathOperator*{\argmaxOPR}{argmax} % aux
	\newcommand{\argmax}[1]{\argmaxOPR_{#1}}
	\newcommand{\dargmax}[1]{\dpl\argmaxOPR_{#1}}
	\DeclareMathOperator*{\argminOPR}{argmin} % aux
	\newcommand{\argmin}[1]{\argminOPR_{#1}}
	\newcommand{\dargmin}[1]{\dpl\argminOPR_{#1}}
	
	%% complex analysis
	\DeclareMathOperator{\re}{Re}
	\DeclareMathOperator{\im}{Im}
	\newcommand{\cis}[1]{\mathe^{#1}}
	\DeclareMathOperator{\Arg}{Arg}
	
% Algebra.
	%% matrices
	\newcommand{\trsp}{^{\mathrm{T}}} % Transpose.
	\newcommand{\ivs}{^{-1}} % Inverse.
	\newcommand{\mat}[1]{\begin{pmatrix}#1\end{pmatrix}} % Matrix.
	\newcommand{\bat}[1]{\begin{bmatrix}#1\end{bmatrix}} % Bracket matrix.
	\newcommand{\dete}[1]{\begin{vmatrix}#1\end{vmatrix}} % Determinant.
	\DeclareMathOperator{\rank}{rank}
	\DeclareMathOperator{\tr}{tr} % Trace.
	\newcommand{\adj}{^{\ast}} % Adjoint or conjugate transpose.
	
	%% rings
	\DeclareMathOperator*{\gcdOPR}{gcd} % aux
	\renewcommand{\gcd}[1]{\gcdOPR\mparen{#1}}
	\DeclareMathOperator*{\lcmOPR}{lcm} % aux
	\newcommand{\lcm}[1]{\lcmOPR\mparen{#1}}
	\newcommand{\divi}[2]{\left.#1\ \middle|\ #2\right.} % Divisible.
	
	%% group theory
	\DeclareMathOperator{\aut}{Aut}
	
% Geometry.
	% \renewcommand{\parallel}{\,/\hspace{-0.2em}/\,} % EM deprecated.
	\DeclareMathOperator{\parallelOPR}{//} % aux
	\renewcommand{\parallel}{\parallelOPR}
	\newcommand{\vct}[1]{\overrightarrow{#1}}
	\newcommand{\x}{u^1}
	\newcommand{\y}{u^2}
	\newcommand{\coo}[1]{u^{#1}}
	\newcommand{\tx}{\tilde{u}^1}
	\newcommand{\ty}{\tilde{u}^2}
	\newcommand{\tco}[1]{\tilde{u}^{#1}}
	\newcommand{\crs}[3]{\Gamma_i^j\ _k} % Christoffel notation of the 2nd kind.
	
% Topology
	\DeclareMathOperator{\Int}{\mathrm{Int}} % Internal.
	
% Probability.
	\newcommand{\prb}[1]{\mathbb{P}\mparen{#1}}
		\newcommand{\bprb}[1]{\mathbb{P}\bparen{#1}}
		\newcommand{\Bprb}[1]{\mathbb{P}\Bparen{#1}}
		\newcommand{\bgprb}[1]{\mathbb{P}\bgparen{#1}}
		\newcommand{\Bgprb}[1]{\mathbb{P}\Bgparen{#1}}
	\DeclareMathOperator{\mep}{\mathbb{E}} % Mathematical expectation.
	\newcommand{\mch}{\mathds{1}} % Characterization function.
	\DeclareMathOperator{\var}{var} % Variance.
	\DeclareMathOperator{\cov}{cov} % Covariance.
	\newcommand{\aseq}{\xlongequal{\text{a. s. }}} % Almost surely equal to.
	\newcommand{\asto}{\xlongrightarrow{\text{a. s. }}} % Almost surely converge to.
	\newcommand{\pbto}{\xlongrightarrow{\mathbb{P}}} % Converge in probability to.
	\newcommand{\lrto}[1][r]{\xlongrightarrow{L_{#1}}} % Converge in L_p to, p == r by default.
	\newcommand{\wkto}{\xlongrightarrow{\text{d}}} % Weakly converge to.
	\newcommand{\iid}{\overset{\text{i.i.d.}}{\sim}} % Independently identically distributed.